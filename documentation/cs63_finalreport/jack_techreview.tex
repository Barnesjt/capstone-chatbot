\subsubsection{Abstract}
This document is technology review of the three essential components, and their available options, that comprise the back-end portion of the NetScaler Chatbot project.
The first component is a bi-directional connection and the messaging pattern it will use.
The second component is the data storage solution the back-end will implement.
The last component is the authentication scheme that will be used to bind users to accounts.
At the conclusion of the document, options for each component will be weighed and a final determination is made.

\subsubsection{Introduction}
This Technology Review directly concerns components of the back-end of the proposed software: NetScaler ChatBot.
Within this document, each component and available options are reviewed.
At the end of the document, within the conclusion, each option is compared within a discussion section.
After a comparison, a final determination about the selected technology/methodology for each option is made.

\paragraph{Project Information}
To ensure the essential operation of the proposed software, NetScaler ChatBot, initial document requirements call for 2 component softwares to be developed in unison.
The first software is a ChatBot, which is hosted on a cloud service and serves at the interface for the user to interact with.
The second software is known as the Relay.
The Relay sits on the Oregon State University firewall, awaiting external requests from the ChatBot. 
In this design pattern, the ChatBot itself operates as the front-end, and the Relay operates as the back-end.

\paragraph{Software Operation Pattern}
In the proposed design of the software, users interface directly with the ChatBot, sending requests in the form of commands.
The command (if valid), along with information about the user, is packaged and sent to the relay.
The relay processes the request against the Citrix NetScaler.
The result is packaged, sent back to the ChatBot, and displayed to the user via a chat message.

\paragraph{Components Reviewed}
Three components and their available options will be reviewed.
The first of these components is the messaging pattern that will be implemented for it's secured, bidirectional connection with the Chatbot.
Available messaging patterns include Reverse Proxy, Message Queue, and Pub Sub.
The second component is the data storage implementation that will be used to store state data.
Available data storage implementations are persistent storage, in-memory data storage, and no external storage (in-application only).
The final component to be reviewed is the authentication scheme that will be used to bind users with accounts.
Potential authentication schemes include: Central Authentication Service (CAS) and Non-Federated.

\subsubsection{Messaging Pattern}
This component of the back-end governs the style of communication between the externally hosted ChatBot and it's purpose designed Relay.
Between the ChatBot and the Relay will be a secure connection.
Any valid command issued by the user to the ChatBot will generate a request that must then be sent to the Relay.
The Relay will then have some amount of work to perform, it must then send a response back to the ChatBot.
This chosen pattern should be reliable, be bi-directional in nature, and have a straight-forward implementation to reduce moving parts.

\paragraph{Reverse Proxy}
A reverse proxy is a type of server that retrieves resources from a server that is hidden from the client, appearing at the originator \cite{rproxy}.
This differs from a standard proxy that relays and imitates the client when requesting resources from additional servers \cite{apacheproxy}.
Due to the nature of proxies, the messaging pattern is naturally bi-directional.
A reverse proxy traditionally operates with HTTP-like requests \cite{rproxy} with each received request receiving a response.
Due to the bidirectional nature of a Reverse Proxy pattern, this is likely to be the easier implementation of the 3 options.
One option for implementation of this messaging pattern would be to use a Node.js package like https-proxy-agent \cite{npmproxy}.

\paragraph{Message Queue}
At the most basic level, "A message queue is a queue of messages sent between applications. It includes a sequence of work objects that are waiting to be processed".
Further, "A message is the data transported between the sender and the receiver application; it's essentially a byte array with some headers at the top" \cite{cloudamqp}.
A message queue isn't traditionally bidirectional, instead operating asynchronously in a one-way direction, without an expected response \cite{cloudamqp}.
By having a queue built asynchronously, messages are processed in the order they are received, and do not interfere with each other.
An existing implementation of a message queue is RabbitMQ, which runs using the AMQP protocol \cite{cloudamqp}.

\paragraph{Pub Sub}
"Pub/sub is shorthand for publish/subscribe messaging, an asynchronous communication method in which messages are exchanged between applications without knowing the identity of the sender or recipient" \cite{pubsubstack}.
This "loose coupling" means that publishers are never aware of the subscribers who receive their messages \cite{pubsubstack}.
Scalability is a priority with the Pub Sub messaging pattern.
Unlike previous options, Pub Sub can be a many-to-many relationship.
As with message queues, a pub sub pattern is not traditionally bidirectional \cite{pubsubstack}.
An existing implementation of a pub sub pattern is PubSubJS \cite{pubsubjs}.

\subsubsection{Data Storage}
Depending on the implementation of other features, a database may be needed to manage data in a more persistent manner.
One example of a potential need for data persistence would be an authentication scheme/pattern that requires storing active tokens.
Additionally, users may need to be matched to specific NetScaler accounts or partition IPs, a persistent database would allow this functionality in a reliable manner.

\paragraph{SQLite}
"SQLite is a software library that provides a relational database management system" \cite{whatissqlite}.
SQLite is a full featured relational database system that operates out of a single file without an additional server process \cite{whatissqlite}.
SQLite is open source, fully cross-platform, and the most widely deployed database in the world \cite{sqlite}.
According to it's own website, "SQLite is very carefully tested prior to every release and has a reputation for being very reliable." \cite{sqlite}

Another advantage of this option is it's persistent and also full ACID-compliant \cite{sqlite}.
This "means all queries and changes are Atomic, Consistent, Isolated, and Durable" \cite{whatissqlite}.
This means that a transaction will either complete fully, or not all.
This prevents conditions that may corrupt the database file.
This is a robust solution in a lightweight, in-application, package.

\paragraph{Redis}
Redis stands for Remote Dictionary Server \cite{redisfaq}.
It is an in-memory external data structure meant to be fast, storing all data in key-value pairs \cite{redisfaq}.
Even though it exists primarily in memory, it can be set up to persist data to a hard disk \cite{redisintro}.
Redis runs as an external process, requiring programs that use it to connect to use a client interface.
Like SQLite, Redis is also open source and supported by a large variety of languages \cite{rediswhatandwhy}.
To use Redis, an instance of it must also be deployed separately.  

\paragraph{Stateless}
This option is a stateless methodology, without reliance on date persistence.
The feasibility of a stateless design is still in question, but is worth exploring as it's the client's preference.
This option would require engineering a methodology to either derive required information, or to create a configuration on the NetScaler that would allow this type of operation.

Given the nature of the NetScaler's Nitro API documentation \cite{nitro}, the best method to explore the feasibility of this option would be to set up a NetScaler Virtual Machine (VM) to test commands against.
However, this is not something the development team has access to currently.
Within the coming weeks, a test environment will be available and the team can then determine the final design of this component

While this option seems promising, and is worth following up on, it's not worth recommending this as the first option in the current phase.
It would be best to remember this option, choose a different method for data storage, and decide later if it can be phased out later for stateless operation.


\subsubsection{Authentication Scheme}
\paragraph{Central Authentication Service}
The first option for binding users to accounts within the NetScaler is to use Oregon State's University (OSU) Federated login.
This system of Federated logins operates using Central Authentication Service (CAS) \cite{osucas}.
Oregon State University provides a CAS that allows for users to authenticate to services using their ONID credentials.
With configuration, the NetScaler can support logging in via OSU's CAS server \cite{nitro}.
Using such a system, only pre-authorized users would be allowed access to NetScaler configurations.
The downside with this is that users would be asked to authenticate when issuing commands.
Additionally, the relay would need some sort of persistent data storage or require re-authentication with each issued command.

\paragraph{Unique Identifier}
The second option for an authentication scheme would be to gather available user information from the ChatBot software.
This user information includes a unique identifier and the user's email \cite{teamscontext}.
Using this unique user information, user's can be tied to specific NetScaler account credentials.

This method requires persistent data storage to bind users to specific accounts.
Additionally, this method has the potential to be less secure, as the application is being asked to maintain credentials to valuable network hardware.
This also brings into question the ability to spoof identity by attempting to communicate with either software outside of a normal Teams environment.
The pros of this implementation would be a streamlined user experience, not requiring re-authentication. 

\subsubsection{Conclusion}

\paragraph{Messaging Pattern}
\subparagraph{Discussion}
The main requirement for the message pattern is that it is bidirectional.
Of the three reviewed options, only a reverse proxy is naturally bidirectional, with each request expecting a response.
Another factor that plays into the decision is the amount of work that will be required with each request.
From the requirements document, user requests will require a single request to be sent to the Citrix NetScaler (none of the specified commands are compound actions).
Knowing this and the small target user base (~25 from the requirements document), we know that actions are unlikely to queue up.
A message queue, or a pub sub structure is likely to be an overly robust solution when a simple one (that's better suited for request-response pairs) would suffice for the task.
\subparagraph{Recommendation}
I recommend a Reverse Proxy messaging pattern for this project.

\paragraph{Data Storage}
\subparagraph{Discussion}
For the data storage component of the back-end, the primary requirement is for a lightweight deployment.
Only two of the options meet this requirement: SQLite and Stateless.
Currently it is difficult to recommend a Stateless design when many questions remain about the operation of Nitro API functions.
While the team awaits availability of a test environment, the best recommendation is for SQLite, which is the client's second choice behind a stateless design.
SQLite provide a robust database known for reliability without reliance on external applications that complicate deployment.

\subparagraph{Recommendation}
I recommend SQLite for persistent data storage.

\paragraph{Authentication Scheme}
\subparagraph{Discussion}
The client's vision for this software is an implementation without additional logins.
However security demands and concerns may override a simplified user experience.
While both implementations are equally feasible, choosing to leverage OSU's existing CAS seems to be the best option from known information.
Using the CAS provides an extra degree of security by abstracting direct management of critical credentials to sensitive and valuable hardware.

\subparagraph{Recommendation}
I recommend using the Central Authentication Service, as provided by Oregon State University.